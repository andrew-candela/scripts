\documentclass[11pt, oneside]{article}   	% use "amsart" instead of "article" for AMSLaTeX format
\usepackage{geometry}                		% See geometry.pdf to learn the layout options. There are lots.
\geometry{letterpaper}                   		% ... or a4paper or a5paper or ... 
%\geometry{landscape}                		% Activate for rotated page geometry
%\usepackage[parfill]{parskip}    		% Activate to begin paragraphs with an empty line rather than an indent
\usepackage{graphicx}				% Use pdf, png, jpg, or eps§ with pdflatex; use eps in DVI mode
								% TeX will automatically convert eps --> pdf in pdflatex		
\usepackage{bm}
\usepackage{amssymb}
\usepackage{amsmath}
\newcommand{\ffrac}[2]{\ensuremath{\frac{\displaystyle #1}{\displaystyle #2}}}
\setlength{\skip\footins}{1cm}

%SetFonts

%SetFonts


\title{Spam Filter Methodology}
\author{Andrew Candela}
%\date{}							% Activate to display a given date or no date

\begin{document}
\maketitle
\newpage
\section*{The Idea}
The goal of this exercise is to take an email or other message and determine if it is spam or not. I would like to create a function $f: m \rightarrow [0,1]$ that acts on a message $m$ and returns a probability that this message is spam. This function will break the given message up into single words, and for each word observe the frequency of spam and nonspam messages that are observed that contain that word. From this we can estimate the probability that the given message is spam.

Loosely speaking, if a message contains many words that are commonly found in a spam message, say 'offer' or 'deal', then intuitively we would want to assign a high spam probability to this message.

Lets say I have a message $m$ composed of a set of words \(\mathcal{W}\). The probability that a message would contain this set of words given that the message is spam (and similarly for a message that is not spam) is \[P(\,W_1 = x_1,\, \ldots \,, W_n = x_n \mid S \,) = \prod_{w_i \in \mathcal{W}}P(\,W_i=x_i\mid S\,)\] assuming that each of the \(W_i\)'s are independent \footnote{The assumption of independence of the \(W_i \)s is a simplifying one, but perhaps not a realistic one. I suppose I could try to cluster words into 'phrases' and then treat each phrase as an event. I would be able to apply this model and would address the concern that my independence assumption is invalid. Let's first get this to work for single words, and then move on from there.}. Here each \(W_i\) is a RV where 
\[
W_i= \begin{cases}
1, & \text{if } w_i \text{ is in the message} \\
0,  & \text{otherwise} \\
\end{cases}
\]
Ultimately what we want is to be able to assign a probability that a message is spam given a vector of words. We want to estimate $P(\,S \mid \bm{W} = \bm{x}\,)$. Remember that we can use Bayes theorem to write:
\begin{align*}
P(\,S\mid \bm{W}=\bm{x}\,) &= \frac 
	{P(\,\bm{W}=\bm{x} \mid S\,)P(\,S\,)} 
	{P(\,\bm{W}=\bm{x}\,)} \\
	&= \frac
	{P(\,\bm{W}=\bm{x} \mid S\,)P(\,S\,)} 
	{P(\,\bm{W}=\bm{x} \mid S\,)P(\,S\,) + P(\,\bm{W}=\bm{x} \mid \neg S\,)P(\,S\,)}  \\
	&= \frac
	{P(\,\bm{W}=\bm{x} \mid S\,)} 
	{P(\,\bm{W}=\bm{x} \mid S\,) + P(\,\bm{W}=\bm{x} \mid \neg S\,)}
\end{align*}
So we can re-write the probability that a given message composed of words $w \in \mathcal{W}$ is spam as:
\begin{equation}
P(\,S\mid \bm{W}=\bm{x}\,) = \ffrac
	{ \prod_{w_i \in \mathcal{W}} P(\,W_i=x_i \mid S\,)}
	{ \prod_{w_i \in \mathcal{W}} P(\,W_i=x_i \mid S\,) + \prod_{w_i \in \mathcal{W}} P(\,W_i=x_i \mid \neg S\,)}
\end{equation}

Assuming we have a training set of emails \(\mathcal{E}\) composed of spam messages \(\mathcal{S}\) and real messages \(\mathcal{R}\). Let's estimate the probability of seeing spam word $w_i$ in a spam message $s \in \mathcal{S}$ as:
\[
P(\,W_i=1 \mid S\,)=\frac{\text{number of spam emails that contain } w_i}{\text{total number of spam emails}} 
\]
or, in fancy terminology:
\begin{equation}
P(\,W_i=1 \mid S\,)=\frac{ \left\vert\left\{ s \in \mathcal{S} : w_i \in s \right\}\right\vert }{\left\vert\left\{ \mathcal{S} \right\}\right\vert}
\end{equation}

\section*{Practical Concerns}
\subsection*{Underflow} In order to determine $\prod_{w_i \in \mathcal{W}} P(\,W_i=x_i \mid S\,)$ and $\prod_{w_i \in \mathcal{W}} P(\,W_i=x_i \mid \neg S\,)$ we will be multiplying a bunch of floating point numbers together. This may cause a problem, so let's do the following. Let's denote $\rho_i$ to be our estimate that the ith word is in a spam message. In order to compute the product of the $\rho_i$'s, we can use:
\begin{equation}
\prod_i \rho_i = exp \left[ \sum_i \ln(\rho_i) \right]
\end{equation}
\subsection*{Smoothing}Imagine your message contains a word $w_k$ that is not in the training set. This is problematic in two ways. If that word is not present in the training set, then our estimate $\left\vert\left\{ s \in \mathcal{S} : w_k \in s \right\}\right\vert$ of $P(\,W_k=1 \mid S\,)$ would be 0. We would similarly estimate $P(\,W_k=1 \mid \neg S\,)$ to be 0. This will cause a division by zero error in equation (1). If $w_k \notin \mathcal{S}$ but $w_k \in \mathcal{R}$ then we will assign zero probability that this message is spam. Ideally, we would want to admit the possibility that this message is spam, and assign a small but nonzero probability to it.

We accomplish this by adding a 'psuedocount' $k$. Let's now define our probabilities as:
\begin{align}
P(\,W_i=1 \mid S\,)&=\frac{\left\vert\left\{ s \in \mathcal{S} : w_i \in s \right\}\right\vert + k}{\left\vert\left\{ \mathcal{S} \right\} \right\vert + 2k} 
\end{align}
 
\section*{Conclusion}
Looks like we have everything we need to start evaluating our emails. Let's look for a training set and get started searching for spam!


\end{document}  